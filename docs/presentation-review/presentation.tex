\documentclass[12pt, t]{beamer}
\usetheme{Warsaw}
\usepackage[utf8]{inputenc}
\usepackage[IL2]{fontenc}
\usepackage[english]{babel}
\usepackage{hyperref}
\usepackage{amsmath}
\usepackage{amsfonts}
\usepackage{amssymb}
\usepackage{graphicx}
\author{Ondrej Mosnáček, Manoja Kumar Das, \\Mmabatho Idah Masemene}
\title{Project review}
\subtitle{Reviewed team: C}
\institute{PV204 - Security Technologies}
\date{May 19, 2015}

\begin{document}

\begin{frame}
  \titlepage
\end{frame}

\begin{frame}{The reviewed project}
  \begin{itemize}
    \item \emph{Tomcat / Web Browser Based Authentication using JavaCard}
    \item \textbf{Original goal:}
      \begin{itemize}
        \item to provide web-based authentication using JavaCard
      \end{itemize}
    \item \textbf{What was actually implemented:}
      \begin{itemize}
        \item a server application that checks if the client sent the string ``Auth'' and sends back the string ``Authorized''
        \item a client Java browser applet that sends the string ``Auth'' to the server
        \item a JavaCard applet for password-based authentication with a very crude PC application
        \item the browser applet does not communicate with the card at all
      \end{itemize}
  \end{itemize}
\end{frame}

\begin{frame}{Design}
  \begin{itemize}
    \item server and card share a master password and counter
    \item client downloads a Java applet from the server
    \item the browser applet retrieves SHA-1(Password XOR Counter) from the card (and the counter is incremented in the card)
    \item the browser applet sends the hash to the server
    \item the server computes the expected hash and increments the counter
    \item the server checks if the recieved hash matches the expected one
  \end{itemize}
\end{frame}

\begin{frame}[allowframebreaks]{Design flaws}
  \begin{enumerate}
    \item \textbf{the crypto}
      \begin{itemize}
        \item SHA-1(Password XOR Counter)
        \item non-standard construct (possibly prone to cryptanalysis)
        \item ``Never design your own crypto!''
        \item SHA-1 is not considered secure nowadays (but this is addressed in the documentation)
        \item a proper password-based KDF should be used instead (or at least HMAC)
      \end{itemize}
    \item \textbf{no secure channel between client and server}
      \begin{itemize}
        \item design does not mandate the use of TLS for client-server communication
        \item an MITM attacker could hijack the authenticated session after the hash has been sent
      \end{itemize}
      
      ~
    \item \textbf{using a Java browser applet}
      \begin{itemize}
        \item Java browser plugins are now deprecated\footnote{\url{https://blogs.oracle.com/java-platform-group/entry/moving_to_a_plugin_free}}
        \item historically suffered from many vulnerabilites
        \item in this case, the applet requires unlimited permissions -- security problem if server gets compromised
      \end{itemize}
  \end{enumerate}
%  \begin{center}
%    \includegraphics[scale=0.3]{no-java.png}
%  \end{center}
\end{frame}

\begin{frame}[fragile]{Implementation flaws}
  \begin{itemize}
    \item \textbf{JavaCard applet vulnerability}
      \begin{itemize}
        \item access to JavaCard is protected by user PIN
        \item the PIN is set using \verb|INS_SETPIN| APDU instruction
        \item this instruction is not authenticated (the old PIN is not required)
        \item attacker can just set the PIN to a new value and authenticate to the card
        \item after authentication, the attacker can pre-generate a sequence of valid hashes
      \end{itemize}
  \end{itemize}
  \begin{center}
    \includegraphics[scale=0.07]{facepalm.jpg}
  \end{center}
\end{frame}

\begin{frame}{Summary}
  \begin{itemize}
    \item original goals not achieved
    \item several flaws in the design
    \item one serious flaw in the implementation
    \item most code was copied from the Internet/study materials; only small parts were modified
    \item some useless functionality was left over from the original code
     \begin{itemize}
       \item AES encryption and RSA signing in the JavaCard applet
       \item a file containing arbitrary sentences to be returned by the server
     \end{itemize}
  \end{itemize}
\end{frame}
\end{document}